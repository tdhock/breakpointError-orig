\documentclass{jsfds} %autoloads graphicx

% latin1 encoding is strongly adviced; 
\usepackage[latin1]{inputenc}
% If you are using utf8 encoding, please note that jsfds.sty and jsfds.cls have
% be converted into utf8 encoding as well
%\usepackage[utf8]{inputenc}
\usepackage[T1]{fontenc}


\usepackage{amsmath,amsthm}
\newtheorem{definition}{Definition}
\newtheorem{lemma}{Lemma}
\usepackage{dtklogos}
\usepackage{url}
\usepackage{natbib}

\startlocaldefs
% Commandes auteurs spécifiques, par exemple :
\newcommand{\fig}[3][H]{
  \begin{figure}[#1]
    \hskip -1cm
    \input{figures/#2}
    \caption{#3}
    \label{fig:#2}
  \end{figure}
}
\newcommand{\figpdf}[3][H]{
  \begin{figure}[#1]
    \hskip -1cm
    \includegraphics[width=\textwidth]{figures/#2}
    \caption{#3}
    \label{fig:#2}
  \end{figure}
}
\newcommand{\tab}[3][H]{
  \begin{table}[#1]
    \centering
    \input{tables/#2}
    \caption{#3}
    \label{tab:#2}
  \end{table}
}
\usepackage{tikz}
\usepackage{float}
%\usepackage{stfloats}
%\usepackage{morefloats}
\renewcommand{\r}{ \mathbf{ r} }
\newcommand{\rileft}[1][i]{\underline r_{#1}}
\newcommand{\riright}[1][i]{\overline r_{#1}}
\DeclareMathOperator*{\argmin}{arg\,min}
\DeclareMathOperator*{\Diag}{Diag}
\DeclareMathOperator*{\TPR}{TPR}
\DeclareMathOperator*{\FPR}{FPR}
\DeclareMathOperator*{\argmax}{arg\,max}
\DeclareMathOperator*{\maximize}{maximize}
\DeclareMathOperator*{\minimize}{minimize}
\newcommand{\RR}{\mathbb{R}}
%\newcommand{\E}{\mbox{\tiny \`eme}}
%\def\1{1\kern-.20em {\rm l}}
\endlocaldefs

\setmainlanguageenglish
\firstpage{1} % à régler par l'éditeur

\begin{document}

% TODO: UNCOMMENT frontmatter for publication!

% \begin{frontmatter}
  
%   \titre{A breakpoint detection error function for segmentation model
%   selection and validation}
%   \titrecourant{A breakpoint detection error function for segmentation model
%   selection and validation}

%   \begin{InfosAuteurs}
%     \auteur{%
%       \prenom{Toby Dylan} \nom{Hocking}%
%       \thanksref{t1}%
%       \contact[label=e1]{toby.hocking@mail.mcgill.ca}%
%     }%
%     \affiliation[t1]{McGill University Department of Human Genetics, 
%       Montr\'eal, Canada.\\
%       \printcontact{e1}%
%     }
%     \enteteauteurs{TD Hocking} %noms courants des auteurs en tête de page
%   \end{InfosAuteurs}

%   \begin{abstract}
%     We consider the multiple breakpoint detection problem, which is
%     concerned with detecting the locations of several distinct changes
%     in a one-dimensional noisy time series. We propose the
%     breakpointError, a function that can be used to evaluate estimated
%     breakpoint locations, given the known locations of true
%     breakpoints. We discuss an application of the breakpointError to
%     calibrating penalty functions for breakpoint detection. Finally,
%     we show how to relax the breakpointError to obtain an annotation
%     error function which can be used more readily in practice on real
%     data. A fast C implementation of an algorithm that computes the
%     breakpointError is available in an R package on R-Forge.
%   \end{abstract}

% \end{frontmatter}

  % infos fournies par l'éditeur, laisser vide
  % \volume{}
  % \numero{}
  % \anneepublication{}
  % \doi{}
 \arxiv{math.PR/00000000} % Si disponible

  
  \begin{motscles}
    \mot{mode d'emploi}%
    \mot{classe du J-SFdS}%
  \end{motscles}

  \begin{motsclesalter}
    \mot{users guide}
    \mot{J-SFdS document class}
  \end{motsclesalter}
  
  \begin{AMSclass}
    \kwd{35L05}
    \kwd{35L70}
  \end{AMSclass}
 
\newpage

\section{Introduction to segmentation models}

The goal of segmentation is to divide a series of data into distinct
segments. Assume there are $P$ distinct positions in the series at
which data could be gathered. Let $\mathcal P= \{1,\dots,P\}$ be the
set of all such positions. For every position $p\in\mathcal P$, we
assume there is some true probability distribution $D_p$. Let $\mathbb
B=\{1,\dots,P-1\}$ be all bases after which a breakpoint is possible.

\begin{definition}
  A \textbf{breakpoint} is any position $p\in\mathbb B$ for which the
  next position does not have the same distribution: $D_p \neq
  D_{p+1}$. 
\end{definition}

We are given a sample of size $d \leq D$ of data $(p_1, y_1), \dots,
(p_d, y_d)$, with positions $p_i\in\mathcal P$ and noisy observations
$y_i\sim D_{p_i}$ for all samples $i\in\{1, \dots d\}$. 

For example, consider the normal distributions and simulated data
shown in Figure~\ref{fig:motivation}. There are $P=500$ distinct
positions, $d=100$ simulated samples, and 2 breakpoints: $\{300,
400\}$.


\begin{figure}[h!]
  \centering
  \includegraphics[width=\textwidth]{figure-motivation}
  \vskip -0.5cm
  \caption{Two signals with the same breakpoints (vertical dashed
    lines) but different distributional changes. Black circles show
    $d=100$ sampled data points drawn from normal distributions
    defined on bases $\mathcal P=\{1, \dots, 500\}$. Horizontal line
    segments and shaded bands show mean $\pm$ one standard
    deviation. The goal of segmentation is to recover the
    distributions and/or breakpoints, using only the sampled data
    points.}
  \label{fig:motivation}
\end{figure}

t corresponding
positions $\mathbf p\in\mathcal X^{d}$, sorted in increasing order
$p_1<\cdots<p_d$.

Without knowing the underlying probability distributions, there is a
minimum of 0 breakpoints ($D_1=\cdots=D_P$) or a maximum of $P-1$
breakpoints ($D_1 \neq \cdots \neq D_P$).


For the simulations we explore later in this chapter, we assume there
is some latent signal $\mu\in\RR^P$. 

This chapter presents two contributions:
\begin{itemize}
\item Given a latent signal $\mu$, we define a precise breakpoint
  detection error function, and discuss its relationship to the
  annotation error defined in Chapter~\ref{chapter:bams}.
\item We use the error function to determine optimal penalties for
  breakpoint detection in databases of simulated signals of varying
  sampling density, noise, and length.
\end{itemize}

In recent years, several authors have developed a theory of minimal
penalties that can be used to accurately recover a signal from noisy
observations \citep{calibration,lebarbier}. These methods do not take
advantage of the breakpoint annotation data, but instead can be used
offline to analyze some assumptions about the signal and the noise of
the data. Typically, these results guarantee recovery of the correct
signal with high probability. However, in this chapter we are more
interested in accurate recovery of the breakpoints than the signal
itself.
% Asymptotically, selecting the model that best reconstructs the
% latent signal is equivalent to selecting the model that recovers the
% most accurate breakpoints. However, with finite sample size, this is
% not necessarily the case. And we show an example of this inconsistency
% in a simulated data set in
% Figure~\ref{fig:variable-density-sigerr}. 
So here I devote several
pages to directly attacking the problem of breakpoint detection rather
than signal recovery.

% To construct optimal penalties for the recovery of breakpoints in
% piecewise constant signals, we must make some assumptions about the
% process that generates the signals. For the application to copy number
% profiles, we are interested in penalties that can be used for signals
% with varying

% \begin{itemize}
% \item Sampling density in bases per probe.
% \item Noise, which is often profile-dependent.
% \item Number of breakpoints, which we expect to be proportional to the
%   size of the chromosome in base pairs.
% \end{itemize}

Starting in Section~\ref{variable_density}, I will explain how to
construct optimal penalties given certain assumptions about the signal
and noise. But we define optimality in terms of breakpoint detection
error, so I will first devote several pages to a precise definition of
the breakpoint detection error.




\begin{figure}
  \centering
  \includegraphics[width=\textwidth]{figure-motivation-modelOnly}
  \vskip -0.5cm
  \caption{Comparing true and estimated distributions is one approach
    to segmentation model selection/evaluation but is not the subject
    of this paper. \textbf{Top 2 panels}: the same reference/true
    signals as in Figure~\ref{fig:motivation}. \textbf{Others}: estimated
    maximum likelihood models for $k\in\{2, \dots, 5\}$ segments.}
  \label{fig:motivation-modelOnly}
\end{figure}

\begin{figure}
  \centering
  \includegraphics[width=\textwidth]{figure-motivation-breaksOnly}
  \vskip -0.5cm
  \caption{Same as Figure~\ref{fig:motivation-modelOnly}, but showing
    breakpoints instead of distributions. This paper proposes to
    compare the true and estimated breakpoints with the
    breakpointError function, which can be used for any reference/true
    distribution and any type of change. }
  \label{fig:motivation-breaksOnly}
\end{figure}

\begin{figure}
  \centering
  \input{figure-motivation-breakpointError}
  \vskip -0.5cm
  \caption{Two signals with the same breakpoints (vertical dashed
    lines) but different distributional changes. Black circles show
    $d=100$ sampled data points drawn from normal distributions defined on
    bases $\mathcal X = \in\{1, \dots, P=500\}$. 
    Horizontal line segments and shaded
    bands show mean $\pm$ one standard deviation. The goal of
    segmentation is to recover the distributions and/or breakpoints,
    using only the sampled data points.}
  \label{fig:motivation-breakpointError}
\end{figure}


The top two panels of Figure~\ref{fig:TODO} show two different
time-varying normal distributions with the same breakpoints: the
left panel shows a change in mean, and the right panel shows a change
in variance. Some simulated data from these distributions are also
shown.  There are several ways to measure the distance between the
true and estimated probability distributions.  cite earth mover
distances \citep{earth-mover}, which is the same as the first
Wasserstein distance.

\subsection{Maximum likelihood segmentation}

We will focus on the cghseg model, which
defines the estimated signal with $k$ segments as
\begin{equation}
\label{eq:yhat^k}
\begin{aligned}
\hat y^k = &\argmin_{x\in\RR^d} &&  ||y - x||^2_2
\\
&\text{subject to} && k-1=\sum_{j=1}^{d-1} 1_{x_j\neq x_{j+1}}.
\end{aligned}
\end{equation}
Note that we can quickly calculate $\hat y^k$ for
$k\in\{1,\dots,k_{\text{max}}\}$ using pruned dynamic programming
\citep{pruned-dp}. We then estimate the breakpoint locations using the mean
\begin{equation}
  \label{eq:breaks_phi}
\phi(\hat y^k,p)
= \big\{
\lfloor 
(p_j+p_{j+1})/2
\rfloor
\text{ for all }j\in\{1,\dots,d-1\}\text{ such that }
\hat y^k_j\neq \hat y^k_{j+1}
\big\}.
\end{equation}
Note that this is a function $\phi:\RR^d\times \mathcal X^d\rightarrow
2^{\mathbb B}$ that gives the positions after which there is a break
in $\hat y^k$. We would like to compare these estimated breakpoints to
the exact set of breakpoints in the simulated signal
\begin{equation}
  \label{eq:breaks_B}
  B = \phi\left(\mu,
\left[
  \begin{array}{ccc}
    1 & \cdots & P
  \end{array}
\right]^\prime
\right)
=
\{j\in\mathbb B:\mu_j\neq\mu_{j+1}\}.
\end{equation}


\newpage

Given some guess of the breakpoint locations $G\subseteq\mathbb B$,
the object of this section is to define a function ${E }(G)$
that quantifies how bad the breakpoint location guess was. We would
like the function ${E }: 2^{\mathbb B}\rightarrow \RR^+$ to
satisfy:

\begin{itemize}
\item \textbf{(correctness)} Guessing exactly right costs nothing: ${E }(B)=0$.
\item \textbf{(precision)} A guess closer to a real breakpoint is less
  costly:\\if $B=\{b\}$ and $0\leq i<j$, then
  ${E }(\{b+i\})\leq{E }(\{b+j\})$ and
  ${E }(\{b-i\})\leq{E }(\{b-j\})$.
% In
%   fact, we will see that when we know the exact breakpoint locations
%   in simulated signals, we can construct a cost function that verifies
%   the strict inequality $<$. In real data, we will use the annotation
%   error which only verifies the weak inequality $\leq$.
\item \textbf{(FP)} False positive breakpoints are
  bad: if $b\in B$ and $g\not\in B$, then ${E }(\{b\}) <
  {E }(\{b,g\})$.
\item \textbf{(FN)} Undiscovered breakpoints are bad:
  $b\in B\Rightarrow{E }(\{b\}) < {E }(\emptyset)$.
\end{itemize}

Keeping these properties in mind, in this section we define 4 such
breakpoint detection error functions $E$:
\begin{itemize}
\item When the latent signal is available in simulations, we can use
  the exact breakpoint locations $B$ to define $E^B_{\text{exact}}$,
  which satisfies all 4 properties.
\item In real data, we do not know the exact breakpoint locations $B$,
  but we can approximate them using regions $\hat R$ that visually
  contain breakpoints. Assuming that there is exactly 1 region in
  $\hat R$ for every breakpoint in $B$, we can define
  $E^{\hat R}_{\text{complete}}$, which also satisfies all 4 properties.
\item In real data, we may not want to assume that the regions $\hat
  R$ contain all breakpoints $B$. So we define $E^{\hat
    R,A}_{\text{incomplete}}$, which only satisfies these properties
  if the annotations $A$ are consistent with the real breakpoints $B$.
\item In real data, an imperfect annotator is making the database of
  regions $\hat R$ and annotations $A$. So we define $E_{01}^{\hat
    R,A}$ to limit the influence of each annotation.
\end{itemize}

We will define each of these breakpoint detection error functions in
the next sections, then compare them in
Figures~\ref{fig:variable-density-sigerr-small} and
\ref{fig:variable-density-sigerr}.
%Section~\ref{sec:incomplete}.

\newpage

\section{Exact breakpoint error for simulated signals}
\label{sec:breakpoint_error}

In this section, we use the exact breakpoint locations $B$ to define a
breakpoint detection error function.

We define the error of a breakpoint location guess $g\in\mathbb
B$ as a function of the closest breakpoint in $B$. So
first we put the breaks in order, by writing them as $B_1<\cdots<
B_n$, with each $B_i\in\mathbb B$. Then, we define a set of intervals
$R_B=\{\r_1,\dots,\r_n\}$ that form a partition of $\mathbb B$. For each
breakpoint $B_i$ we define the region
${\r}_i=[\rileft,\riright]\in\mathbb I \mathbb B$, where $\mathbb
I\mathbb B\subset 2^{\mathbb B}$ denotes the set of all intervals of
$\mathbb B$. We take the notation conventions from the interval
analysis literature \citep{intervals}.

We define the right limit of region $i$ as
\begin{equation}
  \label{eq:R_i}
\riright
=
  \begin{cases}
    P-1 & \text{if } i=n \\
    \lfloor (B_{i+1}+B_i)/2 \rfloor & \text{otherwise}
  \end{cases}
\end{equation}
and
the left limit as
\begin{equation}
  \label{eq:L_i}
  \rileft =
  \begin{cases}
    1 & \text{if } i=1 \\
    \riright[i-1]+1 & \text{otherwise}.
  \end{cases}
\end{equation}

The breakpoints $B_i$ and regions $\r_i$ are labeled for a small
signal in Figure~\ref{fig:exact_imprecision}.

\begin{figure}[b]
  \centering
  \input{figures/breakpoint-error-pieces}
  \caption{For 2 breakpoints $i$, we plot their functions $\ell_i$
    that measure the precision of a guess.
%, and are used to calculate $I_i$ in
 %   Equation~\ref{eq:imprecision}. 
    The signal $\mu\in\RR^{22}$ has 2 breakpoints: $B=\{4,14\}$.}
  \label{fig:exact_imprecision}
\end{figure}


Intuitively, if we observe a breakpoint guess $g\in \r_i$, then its
closest breakpoint is $B_i$. To define the best guess in each region,
we use piecewise linear functions $C_{\underline r,b,\overline r}:\RR\rightarrow[0,1]$
defined as follows:
\begin{equation}
  \label{eq:cLxR}
  C_{\underline r,b,\overline r}(g) =
  \begin{cases}
    0 & \text{if }g=b \\
    (b-g)/(x-\underline r) & \text{if } \underline r< g< b \\
    (g-b)/(\overline r-x) & \text{if } b< g< \overline r\\
    1 & \text{ otherwise}.
  \end{cases}
\end{equation}
For each breakpoint $i$ we measure the precision of a guess
$g\in\mathbb B$ using
\begin{equation}
  \label{eq:ell_i_exact}
  \ell_i(g)=C_{\rileft,B_i,\riright}(g).
\end{equation}
These functions are shown in Figure~\ref{fig:exact_imprecision} for a
small signal with 2 breakpoints.
% We
% will use $\ell_i$ to define the imprecision~$I_i$, one component of
% the exact breakpoint error.

Now, we are ready to define the exact breakpoint error of a set of
guesses $G\subseteq\mathbb B$.
%We will use $i$ for the index of the
%breakpoints $B_i$ and $j$ for the index of the guesses $G_j$. 
First, let $G \cap\r$ be the subset of guesses $G$ that fall in
region~$\r$. 

Then, we define the false negative rate for region $\r$ as 
\begin{equation}
  \label{eq:FN_i}
  \text{FN}(G,\r) = 
  \begin{cases}
    1 & \text{if } G\cap\r = \emptyset\\
    0 & \text{otherwise}
  \end{cases}
\end{equation}
and the false positive rate for region $\r$ as
\begin{equation}
  \label{eq:FP_i}
  \text{FP}(G,\r) =
  \begin{cases}
    0 & \text{if }G\cap\r = \emptyset\\
    |G\cap\r|-1 &\text{otherwise}
  \end{cases}
\end{equation}
and the imprecision of the best guess in region $r$ as
\begin{equation}
  \label{eq:imprecision}
  I(G,\r,\ell) =
  \begin{cases}
    0 & \text{if } G\cap\r = \emptyset\\
    \min_{g\in G\cap\r} \ell(g) & \text{otherwise}.
  \end{cases}
\end{equation}
When there are no breakpoints, we have $B=\emptyset$ and
$R_B=\emptyset$. But we still would like to quantify the false
positives, so let $G\setminus\big( \cup R_B\big) $ be the set of
guesses $G$ outside of the breakpoint regions $R_B$. Finally, we
define the exact error of guess $G$ with respect to the true
breakpoints $B$ as
\begin{equation}
  \label{eq:exact_breakpoint_cost}
  {E }_{\text{exact}}^B(G) =
  \big|G\setminus(\cup R_B)\big|
 + \sum_{i=1}^{|B|}\text{FP}(G,\r_i)+\text{FN}(G,\r_i)+I(G,\r_i,\ell_i).
\end{equation}

\newpage

To calculate the exact breakpoint error, we first sort lists of
$n=|B|$ and $m=|G|$ items. Using the quicksort algorithm, this
requires $O(n\log n + m\log m)$ operations in the average case
\citep{clrs}. Once sorted, the components of the cost can be
calculated in linear time $O(n + m)$. So, overall the calculation of
the error can be accomplished in best case $O(n + m)$, average case
$O(n\log n + m\log m)$ operations.

\newpage
\section{Incomplete annotation error for real data}
\label{sec:incomplete}

In real data, we do not have access to the latent signal $\mu$, nor
the underlying set of breakpoints $B$. However, by
plotting the data, we can easily identify regions that contain
breakpoints by visual inspection, as shown in
Figure~\ref{fig:variable-density-annotation-cost}.

In general, let $A=\{a_1,\dots,a_n\}$ be some annotations and $\hat
R=\{\hat \r_1,\dots,\hat \r_n\}$ be the corresponding regions. For
every annotation~$i$, let $\hat\r_i\subset\mathbb I\mathbb B$ be the
interval that defines the region, and let $a_i\subseteq\{0,1,\dots\}$
be the interval of allowable breakpoint counts in this region. For
example, consider the annotated regions in
Table~\ref{tab:sample_annotations}.

\begin{table}[b!]
  \begin{center}
    \begin{tabular}{ccc}
  $i$ & Annotation $a_i$ & Region $\hat \r_i$\\
\hline
1 & \{0\} & [5,10]\\
2 & \{1\} & [20,30]\\
3 & \{1,2,\dots\} & [40,70]\\
4 & \{0\} & [80,100]
\end{tabular}
  \end{center}
  \caption{Sample annotated regions for a signal sampled on $P=100$ base pairs. 
    An annotation $a_i$ indicates how many breakpoints are allowed in the corresponding region $r_i$.
    There are 0 breaks in bases 5-10 and 80-100.
    There is exactly 1 break in bases 20-30.
    There is at least 1 break in bases 40-70.}
  \label{tab:sample_annotations}
\end{table}




Then, we define the annotation-dependent false positive rate as
\begin{equation}
  \label{eq:FP_hat}
  \hat{ \text{FP}}(G,\r,a) =
    \big( 
|G\cap\r|-\max(a)
\big)_+
\end{equation}
and the annotation-dependent false negative rate as
\begin{equation}
  \label{eq:FN_hat}
  \hat{ \text{FN}}(G,\r,a) =
  \big(
\min(a)-|G\cap\r|
\big)_+.
\end{equation}

So then we define the incomplete annotation error for a set of
regions $R$ with annotations $A$ as
\begin{equation}
  \label{eq:incomplete}
  E^{R,A}_{\text{incomplete}}(G)=
    \sum_{(\r,a)\in(R,A)} \hat{\text{FP}}(G,\r,a) + \hat{\text{FN}}(G,\r,a).
\end{equation}

\newpage 

In the case of analyzing the simulated signals in the top panels of
Figure~\ref{fig:variable-density-annotation-cost}, let us consider the
set of regions $\hat R$ depicted using the red rectangles. Every
region $\hat\r_i\in\hat R$ contains exactly 1 breakpoint
$B_i\in\hat\r_i$, so we have a corresponding set of annotations
$ A$ with $a=\{1\}$ for every annotation $a\in A$. In real data we
will probably only be able to see a subset of the real breakpoints,
but we analyze it here to illustrate the approximation induced by the
annotation process.

Given the set of breakpoint regions $\hat R$, we define $|\hat R|+1$
negative regions
\begin{equation}
  \label{eq:R^0}
  \hat R^0 = \big\{ 
[1,\underline r_1-1],
[\overline r_1+1, \underline r_2-1],
\dots,
[\overline r_{n-1}+1,\underline r_n-1],
[\underline r_n+1,d-1]
\big\},
\end{equation}
as shown in the middle panels of
Figure~\ref{fig:variable-density-annotation-cost}. There is a
corresponding set of annotations $A^0$ with $a=\{0\}$ for every
annotation $a\in A^0$. We will use the complete set of regions $\hat
R\cup \hat R^0$ and annotations $A\cup A^0$ to define the
annotation error $E^{\hat R\cup\hat R^0,A\cup
  A^0}_{\text{incomplete}}$ for models of these simulated signals.


In Figure~\ref{fig:variable-density-sigerr-small}, we plot some model
selection error functions for the 2 simulated signals shown in
Figure~\ref{fig:variable-density-annotation-cost}. It is clear that
the annotation error is a good approximation of the breakpoint error,
and there are several interesting observations to note. 

\newpage

\begin{itemize}
\item \textbf{Signal}: we plot the log squared error of the estimated
  signal $\hat y^k$ with respect to the latent signal $\mu$, defined
  as
  \begin{equation}
    \label{eq:signal_cost}
    E^{\text{signal}}(k) = \log_{10}\left[
\frac 1 d \sum_{i=1}^d(\hat y_i^k - \mu_i)^2
\right].
  \end{equation}
  \begin{itemize}
\item For the signal sampled at 7 bases/probe, the minimum of the
  error identifies the correct model with 7 segments.
\item For the signal sampled at 374 bases/probe, the minimum of the
  error identifies an incorrect model with only 5
  segments.
  \end{itemize}
\item \textbf{Breakpoint}: for both signals, the minimum of the
  breakpoint error identifies the correct model. 
\item \textbf{Annotation}: the annotation error is a close
  approximation of the breakpoint error, and also identifies the
  correct model.
\end{itemize}
\begin{figure}[H]
  \centering
  \input{figures/variable-density-sigerr-small}
  \caption{Model selection error curves for 2 simulated
    signals. Minima are highlighted using circles.
    \protect\\
    \textbf{Signal}: the log squared error $E^{\text{signal}}$ of the
    estimated signal $\hat y^k$ for $k$ segments with respect to the
    latent signal $\mu$.
    \protect\\
    \textbf{Breakpoint}: exact breakpoint error $E^B_{\text{exact}}$.
    %described in Section~\ref{sec:breakpoint_error}.
    \protect\\
    \textbf{Annotation}: incomplete annotation error
    $E^{R,A}_{\text{incomplete}}$.
%for the annotations shown in
 %   Figure~\ref{fig:variable-density-annotation-cost}. 
}
  \label{fig:variable-density-sigerr-small}
\end{figure}

\newpage


\section{Link with breakpoint error using 
complete annotation error}
\label{sec:complete}
It is clear from Figure~\ref{fig:variable-density-sigerr-small} that
the annotation error is a good approximation of the exact breakpoint
error when the annotated regions $\hat R,A$ agree with the real
breakpoints $B$. In this section, we make this intuition precise by
showing exactly how to relax the breakpoint error to obtain the
annotation error. There are two steps:
\begin{itemize}
\item First, we define the complete annotation error by relaxing the
  definition of the exact breakpoint error.
\item Then, we show that the complete annotation error is equivalent
  to the incomplete annotation error when we have a complete set of
  annotations.
\end{itemize}

\begin{figure}[b!]
  \centering
\includegraphics[width=\linewidth]{figures/variable-density-annotation-cost}
%\input{figures/variable-density-annotation-cost}
\vskip -0.1in
  \caption{\textbf{Top}: simulated noisy
  signals (black) with their latent signals $\mu$ (blue) and
  visually-determined breakpoint annotations $\hat R$
  (red). 
\protect\\
\textbf{Middle}: negative regions $\hat R^0$ constructed
  using (\ref{eq:R^0}).
\protect\\
\textbf{Bottom}: breakpoint detection
  imprecision curves for the breakpoint error $\ell_i$
  (\ref{eq:imprecision}) and the annotation error $\hat
  \ell_i$ (\ref{eq:hat_ell_i}).}
\label{fig:variable-density-annotation-cost}
\end{figure}

\subsection{The complete annotation error}

We define the complete annotation error as a relaxation of the exact
breakpoint error. So first, let us recall the definition of the exact
breakpoint error from (\ref{eq:exact_breakpoint_cost}):
$$
  {E }_{\text{exact}}^B(G) =
  \big|G\setminus(\cup R_B)\big|
 + \sum_{i=1}^{|B|}\text{FP}(G,\r_i)+\text{FN}(G,\r_i)+I(G,\r_i,\ell_i).
$$
To define the complete annotation error, we perform two relaxations:
\begin{itemize}
\item Instead of using (\ref{eq:R_i}) and (\ref{eq:L_i}) to
  determine a breakpoint region $\r_i$, we use the region $\hat\r_i$
  determined by visual inspection.
\item Rather than the piecewise linear imprecision $\ell_i$, we use
  the zero-one imprecision $\hat \ell_i$:
\begin{equation}
  \label{eq:hat_ell_i}
  \hat\ell_i(g) = 1_{g\not\in \hat \r_i}.
\end{equation}
We show this relaxation by ploting the imprecision functions $\ell_i$
and $\hat \ell_i$ in the bottom panels of
Figure~\ref{fig:variable-density-annotation-cost}.
\end{itemize}
So, performing these two relaxations results in the following
definition of the complete annotation error:
\begin{eqnarray}
  E_{\text{complete}}^{\hat R} (G)
&=&  \Big|G\setminus(\cup \hat R)\Big| \nonumber
 + \sum_{i=1}^{|\hat R|}\text{FP}(G,\hat\r_i)+\text{FN}(G,\hat\r_i)+I(G,\hat\r_i,\hat\ell_i)\\
&=&  \Big|G\setminus(\cup \hat R)\Big|
 + \sum_{\hat\r\in\hat R}\text{FP}(G,\hat\r)+\text{FN}(G,\hat\r)
\end{eqnarray}
It is clear that $E^{\hat R}_{\text{complete}}$ depends on the
annotations only through their regions. In particular, the annotated
breakpoint counts $a_i=\{1\}$ are not used in this definition, since
we assumed that each region $\hat \r_i$ contains exactly 1
break. Also, since we used the zero-one loss for $\hat\ell_i$, the
imprecision function $I$ is always zero.



% First, let us enlarge the set of annotated regions so that it covers
% all the possible breakpoints $\mathbb B$. 
\subsection{Equivalence of complete and incomplete annotation error}



To see the connection between the complete and incomplete annotation
error functions, note that
\begin{eqnarray}
\nonumber  \hat{\text{FN}}(G,\r,\{1\}) 
&=&   \big(
1-|G\cap\r|
\big)_+\\
&=&
\text{FN}(G,\r), \label{eq:fn-hat}
\end{eqnarray}
and
\begin{eqnarray}
\nonumber
  \hat{\text{FP}}(G,\r,\{1\}) 
&=&
\big( 
|G\cap\r|-1
\big)_+\\
% &=&
% \begin{cases}
%   0 & \text{if }|H(G,r)|\leq 1\\
%   |H(G,r)| - 1 & \text{if }|H(G,r)|\geq 1
% \end{cases}\\
&=& \text{FP}(G,\r). \label{eq:fp-hat}
\end{eqnarray}
For the complete annotation error we quantified the false positive
rate of the breakpoints that fall outside of the breakpoint regions
$\hat R$ using $G\setminus(\cup \hat R)$. For the incomplete
annotation error, we instead created a set of 0-breakpoint annotations
$\hat R^0$ for this purpose. Note that by construction of the negative
regions in (\ref{eq:R^0}), we have 
\begin{equation}
  \label{eq:Rcomplement}
  G\setminus \left(\cup \hat R\right)
 = 
G\cap\left(\cup \hat R^0 \right),
\end{equation}
or in words, the guesses outside of the breakpoint regions $\hat R$
are in the negative regions $\hat R^0$. So using
(\ref{eq:Rcomplement}), we have
\begin{eqnarray}
  \hat{\text{FP}}(G,(\cup \hat R^0),\{0\})
&=& 
|G\cap(
  \cup \hat R^0
)|\nonumber
\\
&=&
|G\setminus(\cup\hat R)|,\label{eq:fp-outside}
\end{eqnarray}
which is the first component of the complete annotation loss.



Recall that $\hat R,A$ represent annotated regions that each contain
exactly 1 breakpoint, and $\hat R^0,A^0$ contain no breakpoints. So
using (\ref{eq:fn-hat}), (\ref{eq:fp-hat}), and
(\ref{eq:fp-outside}), we can show that the incomplete annotation error
is equivalent to the complete error in this sense:
\begin{eqnarray}
  E_{\text{incomplete}}^{\hat R\cup \hat R^0,A\cup A^0}(G)
&=&
\sum_{\r\in \hat R^0} \hat{\text{FP}}(G,\r,\{0\})\nonumber 
+
\sum_{\r\in\hat R} \hat{\text{FP}}(G,\r,\{1\}) + \hat{\text{FN}}(G,\r,\{1\})
 \\
&=&
 \hat{\text{FP}}(G,\cup \hat R^0,\{0\})\nonumber 
+
\sum_{\r\in\hat R} \hat{\text{FP}}(G,\r,\{1\}) + \hat{\text{FN}}(G,\r,\{1\})
 \\
&=&
|G\setminus(\cup\hat R)|
+\nonumber
\sum_{\r\in\hat R} {\text{FP}}(G,\r) + {\text{FN}}(G,\r)\\
&=&
E^{\hat R}_{\text{complete}}(G).
\end{eqnarray}

So in fact the incomplete annotation error is equivalent to the
complete error when the annotated regions $\hat R$ each contain
exactly 1 breakpoint. But we call this the incomplete error since it
is also well-defined for arbitrary sets of regions $\hat R$ and
annotations $A$.

\section{Zero-one annotation error}
\label{sec:zero-one}
The initial method that we used to quantify the breakpoint error on
the neuroblastoma data set in Chapter~\ref{chapter:bams} was the
zero-one loss in (\ref{eq:1error}). In this section, we show
that the zero-one loss is a thresholded version of the incomplete
annotation error function.

First, let us define the zero-one thresholding function
$t:\mathbb Z^+\rightarrow\mathbb Z^+$ as
\begin{equation}
  \label{eq:thresholding}
  t(x)=1_{x\neq 0} =
  \begin{cases}
    1 & \text{if }x\neq 0\\
    0 & \text{otherwise}.
  \end{cases}
\end{equation}

The idea of thresholding is to limit the error that any one annotation
can induce. So instead of counting incorrect breakpoint guesses, we count
incorrect annotations. We define the zero-one annotation error as
\begin{eqnarray}
  \label{eq:ann01err}
  E_{01}^{\hat R,A}(G)
&=&\nonumber
 \sum_{(\r,a)\in(\hat R,A)} 
t\left[\hat{\text{FP}}(G,\r,a)\right]+
t\left[\hat{\text{FN}}(G,\r,a)\right]\\
&=&\nonumber
 \sum_{(\r,a)\in(\hat R,A)} 
1_{|G\cap\r|>\max(a)}+
1_{|G\cap \r|<\min(a)}\\
&=&
 \sum_{(\r,a)\in(\hat R,A)} 
1_{|G\cap\r|\not\in a}.
\end{eqnarray}

\newpage 

\section{Comparing annotation error functions}

In practice, we have few annotated regions per signal in real data. In
Figure~\ref{fig:variable-density-sigerr}, we show how the annotation
error is degraded as we remove annotations. In particular, it is clear
that using the thresholded zero-one annotation error significantly
degrades the approximation of the FP curve. Nevertheless, it is worth
noting that minimum of the zero-one error still uniquely identifies
the correct model with 7 segments. Even after removing many
annotations, the minimum error still identifies the correct
model, but not uniquely.

% \begin{itemize}
% \item \textbf{Complete}: the annotation error is a
%   close approximation of the breakpoint error, and its minimum also
%   identifies the correct model.
% \item \textbf{Zero-one}: thresholding degrades the approximation of FP
%   on models with many breakpoints. However, the minimum error still
%   identifies the correct model.
% \item \textbf{Incomplete}: removing half of the annotations further degrades
%   the approximation.
%   \begin{itemize}
% \item However, for the signal sampled at 374 bases/probe, the
%   minimum annotation error still uniquely identifies the correct
%   model. 
% \item For the signal sampled at 7 bases/probe, the
%   $\hat{\text{FP}}$ is a poor approximation of $\text{FP}$ because
%   there are very large regions with no breakpoints and no annotations.
%   \end{itemize}
% \item \textbf{Positive}: the annotation error is formed using only 3
%   annotated breakpoint regions. The minimum annotation error
%   identifies the correct model, but not uniquely.
% \end{itemize}




\begin{figure}[H]
  %\hspace{-1.5cm}
  \input{figures/variable-density-sigerr}
  \caption{Comparison of annotation error functions as the set of
    annotations changes. Minima are highlighted using circles.
    \protect\\
    \textbf{Complete}: annotation error
    $E_{\text{incomplete}}^{\hat R\cup \hat R^0,A\cup A^0}$ 
    for a complete set of 6
    positive and 7 negative annotations.
    \protect\\
    \textbf{Zero-one}: zero-one annotation error $E_{01}^{
      \hat R\cup \hat R^0,A\cup A^0}$
    for the complete annotations.
    \protect\\
    \textbf{Incomplete}: zero-one annotation error $E_{01}^{\hat R,A}$
    for 3 positive and 4 negative annotations.
    \protect\\
    \textbf{Positive}: zero-one annotation error $E_{01}^{\hat R,A}$
    for 3 positive annotations.}
  \label{fig:variable-density-sigerr}
\end{figure}

\newpage

\section{Sampling density normalization}
\label{variable_density}
Having properly defined how to compute the breakpoint detection error
in the previous sections, we now use it to derive several results
about optimal penalties for breakpoint detection. We start by
considering a penalty that is invariant to sampling density. First, we
will present an empirical analysis of several simulated signals using
the breakpoint error. Then, we will discuss the relationship of our
results to relevant theoretical results.

In real array CGH data, the sampling density of probes along the
genome is not uniform across samples. In fact, we see a sampling
density between 40 and 4400 kilobases per probe in the neuroblastoma
data set.

So to construct a penalty that can best adapt to this variation, we
analyze the following simulation. We create a latent signal
$\mu\in\RR^P$ over $P=70000$ base pairs, with breakpoints every 10000
base pairs, shown as the blue line in
Figure~\ref{fig:variable-density-signals}. Then, we define a signal
sample size $d_i\in\{70,\dots,70000\}$ for every noisy signal
$i\in\{1,\dots,n\}$. Let $y_i\in\RR^{d_i}$ be noisy signal $i$,
sampled at positions $p_i\in\mathcal X^{d_i}$, with
$p_{i1}<\cdots<p_{i,d_i}$. We sample every probe $j$ from the
$y_{ij}\sim N(\mu_{p_{ij}},1)$ distribution. These samples are shown
as the black points in Figure~\ref{fig:variable-density-signals}.

We would like to learn some model complexity parameter $\lambda$ on
the first noisy signal, and use it for accurate breakpoint detection
on the second noisy signal. In other words, we are looking for a model
selection criterion which is invariant to sampling density. 

\begin{figure}[H]
\includegraphics[width=\linewidth]{figures/variable-density-signals}
%\input{figures/variable-density-signals}
\vskip -0.3cm
  \caption{Two noisy signals (black) sampled from
  a latent piecewise constant signal (blue). Note that these are the
  same signals that appear in
  Figure~\ref{fig:variable-density-annotation-cost}.}
\label{fig:variable-density-signals}
\end{figure}

\newpage

To attack this problem, we proceed as follows. For every signal $i$,
we use pruned dynamic programming to calculate the maximum likelihood
estimator $\hat y^k_i\in\RR^{d_i}$, for several model sizes
$k\in\{1,\dots,k_{\text{max}}\}$ \citep{pruned-dp}. Then, we define
the model selection criteria
\begin{equation}
  \label{eq:kstar_density}
  k^\alpha_i(\lambda) =\argmin_k \lambda k d_i^\alpha + 
  ||y_i-\hat y^k_i||_2^2.
\end{equation}
Each of these is a function $k_i^\alpha:\RR^+\rightarrow
\{1,\dots,k_{\text{max}}\}$ that takes a model complexity tradeoff
parameter $\lambda$ and returns the optimal number of segments for
signal $i$. The goal is to find a penalty exponent $\alpha\in\RR$ that
lets us generalize $\lambda$ between different signals $i$.


To quantify the accuracy of a segmentation for signal $i$, let
$\text{BErr}_i(k)$ be the breakpoint detection error of the model with
$k$ segments. This is a function
$\text{BErr}_i:\{1,\dots,k_{\text{max}}\}\rightarrow\RR^+$, and in
real data this corresponds to the breakpoint annotation error
$E_{01}^{\hat R,A}\left[\phi(\hat y^k_i,p_i)\right]$. But in
these simulations we can calculate the more precise measure
\begin{equation}
  \label{eq:berr}
  \text{BErr}_i(k) = E_{\text{exact}}^{B}\left[
\phi(\hat y_i^k,p_i)
\right].
\end{equation}
where $B$ is the set of real breakpoints in the latent
signal $\mu$.

In Figure~\ref{fig:variable-density-berr-k}, we plot $\text{BErr}_i$
for the 2 simulated signals $i$ shown previously.  As expected, the
model recovers more accurate breakpoints from the signal sampled at a
higher density.  

\fig{variable-density-berr-k}{Exact breakpoint error
  $\text{BErr}_i(k)$ for two signals $i$ and several cghseg model
  sizes $k$. Note that these are the same error curves that appear
in the Breakpoint panels of Figure~\ref{fig:variable-density-sigerr-small}.}

\newpage

Now, let us define the penalized
model breakpoint error $E^\alpha_i:\RR^+\rightarrow\RR^+$ as
\begin{equation}
  \label{eq:lerr}
E^\alpha_i(\lambda) = \text{BErr}_i\left[
k^\alpha_i(\lambda)
\right].
\end{equation}
In Figure~\ref{fig:variable-density-berr}, we plot these functions for the
two signals $i$ shown previously, and for several penalty exponents $\alpha$.

The dots in Figure~\ref{fig:variable-density-berr} show the optimal
$\lambda$ found by minimizing the penalized model breakpoint detection
error:
\begin{equation}
  \label{eq:lambda_hat}
  \hat \lambda^\alpha_i = \argmin_{\lambda\in\RR^+}  E^\alpha_i(\lambda)
\end{equation}

Figure~\ref{fig:variable-density-berr} suggests that $\alpha\approx1/2$
defines a penalty with aligned error curves, which will result in
$\hat \lambda_i^\alpha$ values that can be generalized between
profiles. 
%Next, we will attempt to determine the optimal $\alpha$
%value.

\fig{variable-density-berr}{Model selection error curves
  $E_i^\alpha(\lambda)$ for 2 signals $i$ and several exponents
  $\alpha$. The penalty contains a term for the number of points sampled $d_i^\alpha$.}

\newpage

Now, we are ready to define 2 quantities that will be able to help us
choose an optimal penalty exponent $\alpha$.

First, let us consider the training error over the entire database:
\begin{equation}
  \label{eq:lerr_train}
  E^\alpha(\lambda) = \sum_{i=1}^n E_i^\alpha(\lambda),
\end{equation}
and we define the minimal value of this function as
\begin{equation}
  \label{eq:lerr_train_min}
  E^*(\alpha) = \min_\lambda E^\alpha(\lambda).
\end{equation}
In Figure~\ref{fig:variable-density-error-train}, we plot these
training error functions $E^\alpha$ and their minimal values $E^*$ for
several values of $\alpha$. It is clear that the minimum training
error is found for some penalty exponent $\alpha$ near 1/2, and we
would like to find the precise $\alpha$ that results in the lowest
possible minimum $E^*(\alpha)$.

\fig{variable-density-error-train}{Training error functions $E^\alpha$
  in black and their minimal values $E^*(\alpha)$ in red. The penalty
  contains a term for the number of points sampled $d_i^\alpha$.}

\newpage

We also consider the test error over all pairs of signals when
training on one and testing on another:
\begin{equation}
  \label{eq:lerr_test}
  \text{TestErr}(\alpha) = 
\sum_{i\neq j} E^\alpha_i(\hat \lambda_j^\alpha).
\end{equation}

In Figure~\ref{fig:variable-density-error-alpha}, we plot $E^*$ and
TestErr for a grid of $\alpha$ values.
It is clear that the optimal penalty is given by $\alpha=1/2$. This
corresponds to the following model selection criterion which is
invariant to sampling density:
\begin{equation}
  \label{eq:var_dens_opt_pen}
  k_i(\lambda) = \argmin_k \lambda k \sqrt{d_i}+||y_i-\hat y_i^k||^2_2
\end{equation}

\fig{variable-density-error-alpha}{Train and test breakpoint detection
  error as a function of penalty exponent $\alpha$. The penalty
  contains a term for the number of points sampled $d_i^\alpha$. Mean
  error is drawn as a black line, with one standard deviation shown as
  a grey band.}

\newpage

As explained by \citet{sylvain-survey}, a model selection procedure
can be either efficient or consistent. An efficient procedure for
model estimation accurately recovers the latent signal, whereas a
consistent procedure for model identification accurately recovers the
breakpoints. Since we consider the breakpoint detection error, we are
attempting to construct a consistent penalty, not an efficient
penalty.

In general terms, the fact that we find a nonzero exponent $\alpha$
for our $d_i^\alpha$ penalty term agrees with other results. In
particular, \citet{vfold} proposed an optimal procedure to select model
complexity parameters in cross-validation by normalizing by the sample
size $d_i$. 
% This is in
% intuitive agreement with our result that we need to normalize by the
% sample size $d_i$, although they do not use the square root term
% $\sqrt{d_i}$.

The $\sqrt{d_i}$ term that we find here using simulations is in
agreement with \citet{aurelie}, who used finite sample model selection
theory to find a $\sqrt{d_i}$ term in a penalty optimal for
clustering.

When theoretically deriving an efficient penalty for segmentation
model estimation in the non-asymptotic setting, \citet{lebarbier}
obtained a $\log d_i$ term. This contrasts our result, which examines
the identification problem using the breakpoint error and obtains a
$\sqrt{d_i}$ term. But in fact this is in agreement with classical
results that AIC underpenalizes with respect to the BIC, as shown in
Table~\ref{tab:AIC-BIC}.

\begin{table}[H]
  \centering
  \begin{tabular}{cc|cc}
     Estimation & Penalty & Identification & Penalty \\
     Model & Term & Model & Term\\
     \hline
     AIC & 2 & BIC & $\log d_i$\\
     Lebarbier & $\log d_i$ & This work & $\sqrt{d_i}$\\
  \end{tabular}
  \caption{Comparing our results with Lebarbier, 
in the context of classical results involving AIC and BIC. 
The BIC is designed for model identification and penalizes more than the AIC.
Likewise, our penalty examines model identification using the breakpoint
detection error, and penalizes more than the efficient penalty proposed
by Lebarbier.}
  \label{tab:AIC-BIC}
\end{table}

\newpage

\section{Scale normalization}
\label{variable_scale}
In real array CGH data, the signal variance is not the same across
samples. In fact, using the variance estimate proposed below in
(\ref{eq:med_abs_diff}), over all the profiles in the
neuroblastoma data set, we see a range of values between 0.029 and
0.371.

So we analyze the following simulation to construct a penalty that is
invariant to the scale of the data. First, we define a theoretical
signal $\mu\in\RR^{700}$ with breakpoints every 100 bases, shown as
the blue line for 2 signals in
Figure~\ref{fig:variable-scale-signals}.

Then, we generate a signal $y_i\in\RR^{d_i}$ by adding standard normal
noise to the signal, and multiplying by a scale factor
$\sigma\in\{1,10,100,1000\}$. The goal is to define a penalty
invariant to this scaling.
For each signal $i$, we estimate its
variance using the robust estimator
\begin{equation}
  \label{eq:med_abs_diff}
  \hat s_i = \text{Median}_{j=1}^{d_i-1}
\left(|y_{ij}-y_{i,j+1}|\right),
\end{equation}
and then define the optimal number of segments as
\begin{equation}
  \label{eq:kstar_shat}
  k_i^\alpha(\lambda) = \argmin_k \lambda k \hat s_i^\alpha 
+ ||y_i - \hat y_i^k||^2_2,
\end{equation}
where $\hat y_i^k$ is the least squares segmentation with $k$
segments.

\fig{variable-scale-signals}{Simulated noisy signal (black) and latent
  signal (blue) for 2 different scales.}

\newpage
Using the type of analysis in the previous section, we saw that
minimum breakpoint detection error is achieved when we choose a
penalty exponent $\alpha$ that results in aligned annotation error
curves $E_i^\alpha$. For two signals $i\neq j$, this implies the
condition
\begin{eqnarray}
  \label{eq:equal_err}
%  E_i^\alpha(\lambda)&=&E_j^\alpha(\lambda)\\
  \text{BErr}_i[k^\alpha_i(\lambda)]&=&  
  \text{BErr}_j[k^\alpha_j(\lambda)].
\end{eqnarray}
Since the assumption of our simulation model is that signals $i$ and
$j$ are identical up to a scaling, the breakpoints are the same and so
$\text{BErr}_i=\text{BErr}_j$. Thus we simply need to find the 
exponent $\alpha$ such that $k^\alpha_i(\lambda)=k^\alpha_j(\lambda)$.

It is easy to see why $\alpha=2$ is the optimal exponent. First, let
$y_1,y_2$ be two signals which are equivalent up to a scaling factor:
$y_2=\sigma y_1$. This implies two interesting properties:
\begin{itemize}
\item \textbf{(a)} estimated models are the same up to a scaling
  factor: $\hat y_2^k = \hat y_1^k\sigma$ for all model sizes $k$.
\item \textbf{(b)} variance estimates are the same up to a scaling factor:
  $\hat s_2 = \hat s_1 \sigma$.
\end{itemize}
From the defintion of $k^\alpha_i$ in (\ref{eq:kstar_shat}) it
is clear that
\begin{eqnarray*}
  \label{eq:k2=k1}
  k^0_2(\lambda) 
&=& \argmin_k \lambda k + ||y_2 - \hat y_2^k||_2^2\\
&=&\argmin_k \lambda k + ||\sigma y_1 - \sigma \hat y_1^k||_2^2 
\ \ \ \textbf{ (a)}\\
&=&\argmin_k \lambda k + \sigma^2||y_1 - \hat y_1^k||_2^2\\
&=&\argmin_k \lambda\sigma^{-2} k+\sigma^2||y_1 - \hat y_1^k||_2^2\\
&=& k^0_1(\lambda \sigma^{-2}).
\end{eqnarray*}
We use this fact to rewrite the equality of the model selection curves
as follows:
\begin{eqnarray*}
  k_1^\alpha(\lambda) &=& k_2^\alpha(\lambda) \\
k_1^0(\lambda \hat s_1^\alpha) &=& k_2^0(\lambda \hat s_2^\alpha)\\
&=& k^0_1(\lambda\hat s_2^\alpha\sigma^{-2})\\
&=&k_1^0\big(\lambda(\hat s_1\sigma^\alpha)\sigma^{-2}\big)\ \ \ \textbf{(b)}\\
&=&k_1^0(\lambda\hat s_1^\alpha \sigma^{\alpha-2}).
\end{eqnarray*}
This implies that $\alpha=2$ must be chosen in order for the model
selection curves to align $k_1^\alpha(\lambda)=k_2^\alpha(\lambda)$.


\newpage

We now proceed with the same kind of empirical analysis that we used
to establish an optimal penalty for sampling density. In
Figure~\ref{fig:variable-scale-berr}, we plot the penalized model
breakpoint detection error $E_i^\alpha$ and the optimal $\hat
\lambda_i^\alpha$ for two signals $i$ and several $\alpha$ values.
This plot suggests that a value of $\alpha=2$ defines a penalty that
will allow generalization of $\hat \lambda^\alpha_i$ values between
signals $i$.

\fig{variable-scale-berr}{Model breakpoint detection error functions
  $E_i^\alpha$ (lines) and optimal $\hat \lambda_i^\alpha$ (points)
  for several penalty exponents $\alpha$ and 2 signals $i$ of
  different scale. The penalty contains a term for the variance
  estimate $\hat s_i^\alpha$.}

\newpage

In Figure~\ref{fig:variable-scale-error-alpha}, we plot the train
error $E^*$ and TestErr as functions of $\alpha$.  In agreement with
our earlier analysis, these simulations clearly indicate that
$\alpha=2$ is optimal. Thus, we conclude that the following model
selection criterion is invariant to scaling:
\begin{equation}
  \label{eq:kstar_shat_opt}
  k_i(\lambda) = \argmin_k \lambda k \hat s_i^2 + ||y_i-\hat y_i^k||^2_2.
\end{equation}
This result agrees with the theoretical results of \citet{lebarbier},
who considered constructing an optimal penalty for breakpoint
detection with respect to the square loss.


\fig{variable-scale-error-alpha}{Train and test error curves as a
  function of penalty exponent $\alpha$, for signals of variable
  scale. The penalty contains a term for the variance estimate $\hat
  s_i^\alpha$.}



\newpage
\section{Signal length normalization}
\label{variable_size}
In real array CGH data, we need to analyze chromosomes of varying length
in base pairs. For example, human chromosome 1 is the largest at about
250 mega base pairs, and chromosome 22 is the smallest with only about
36 mega base pairs. But we expect that the number of breakpoints is
proportional to the length of the chromosome in base pairs, and we would
like to design a model selection criterion that is invariant to the
signal length.

So we consider the following simulation where we fix the number of
points sampled at $d=2000$ and vary the length of the signal
sampled. In Figure~\ref{fig:variable-breaks-constant-size}, we show
samples of 2 different lengths $l_i$, for the same latent signal
$\mu$.

\figpdf{variable-breaks-constant-size}{Samples of 2 different lengths
  $l_i$ but constant number of points $d=2000$.}

\newpage

For each signal $i$, we define the penalty
\begin{equation}
  \label{eq:kstar_length}
  k_i^\alpha(\lambda) = \argmin_k \lambda k l_i^\alpha 
+ ||y_i - \hat y_i^k||^2_2,
\end{equation}
where $l_i$ is the length of the signal in base pairs. The goal will
be to find an $\alpha$ that can be used for signals of varying length.


In Figure~\ref{fig:variable-breaks-constant-size-berr}, we show
the breakpoint detection error curves for two signals and several
penalty exponents $\alpha$.
These curves seem to align when $\alpha=-1/2$.

\fig{variable-breaks-constant-size-berr}{Breakpoint detection error
  curves for several penalty exponents $\alpha$ and 2 samples of
  varying length in base pairs $l_i$. The penalty contains a term
  $l_i^\alpha$.}

\newpage

In Figure~\ref{fig:variable-breaks-constant-size-alpha}, we plot the
train and test error curves over the entire set of simulated signals.
These curves indicate minimal breakpoint detection error at
$\alpha=-1/2$, corresponding to the following penalty:
\begin{equation}
  \label{eq:kstar_length_opt}
  k_i(\lambda) = \argmin_k \frac{\lambda k}{\sqrt{l_i}}
  + ||y_i-\hat y_i^k||^2_2.
\end{equation}




\fig{variable-breaks-constant-size-alpha}{Train and test error curves
  for signals of different length in base pairs $l_i$. The penalty
  contains a term
  $l_i^\alpha$.}

\newpage

Interestingly, the $1/\sqrt{l_i}$ term that we obtain here is in good
agreement with our previous result that the optimal penalty for
variable sampling density $d_i$ should have a $\sqrt{d_i}$ term. In
particular, we can re-parameterize the problem to be in terms of the
number of points sampled per segment $\rho_i=d_i/k_i$. In
Section~\ref{variable_density} we held $k_i$ constant but in this
section we hold $d_i$ constant. In both cases we have a penalty with a
$\sqrt{\rho_i}=\sqrt{d_i/k_i}$ term.

However, we do not know the number of segments $k_i$ in advance. But
we supposed that the number of segments is proportional to the number
of base pairs $l_i$, so we can use that in the penalty. This suggests
a penalty that takes the form of $\sqrt{d_i/l_i}$. So in the next
section, we confirm that this intuition works for constructing an
optimal penalty.


\newpage
\section{Combining normalizations}
\label{combining_penalties}
In this section, we show that we can combine the results of the
previous sections to create composite invariant penalties. In
particular, to normalize for sampling density $d_i$ and length in base
pairs $l_i$, we need $\sqrt{d_i}$ and $1/\sqrt{l_i}$ terms in the
penalty, respectively. This suggests that when considering variable
$d_i$ and $l_i$, we need a $\sqrt{d_i/l_i}$ term in the penalty, and
in this section we show that this intuitive construction results in an
optimal penalty.

In Figure~\ref{fig:variable-size-signals}, we plot 2 signals with
different number of points $d_i$ and length in base pairs $l_i$. We
would like to find a penalty that allows us to generalize model
complexity tradeoff parameters $\lambda$ between these signals.

For each signal $i$, we define the penalty
\begin{equation}
  \label{eq:kstar_composite}
  k_i^\alpha(\lambda) = \argmin_k \lambda k l_i^\alpha \sqrt{d_i}
  + ||y_i - \hat y_i^k||^2_2,
\end{equation}
where $l_i$ is the signal length in base pairs and $d_i$ is the number
of points sampled. We will attempt to determine an $\alpha$ that
allows accurate breakpoint detection in signals of varying length and
number of points sampled. Based on the result in
Section~\ref{variable_size}, we expect to find $\alpha=-1/2$.

\begin{figure}[H]
  \centering
\includegraphics[width=\linewidth]{figures/variable-size-signals}
  \caption{Two signals with a different number of
    points $d_i$ and length in base pairs $l_i$.}
\label{fig:variable-size-signals}
\end{figure}

\newpage

In Figure~\ref{fig:variable-size-berr}, we plot the breakpoint error
functions $E_i^\alpha$ for 2 signals $i$ and several exponents
$\alpha$. The curves seem to align when $\alpha=-1/2$.

\fig{variable-size-berr}{Breakpoint error functions $E_i^\alpha$ for
  several exponents $\alpha$ and 2 signals $i$ of varying number of
  points $d_i$ and length in base pairs $l_i$. The penalty contains a
  term $l_i^\alpha\sqrt{d_i}$.}

In Figure~\ref{fig:variable-size-error-alpha}, we plot the train and
test error as a function of penalty exponent $\alpha$. This analysis
suggests that the optimal exponent is $\alpha=-1/2$, as expected from
our previous analysis in Section~\ref{variable_size}.

\fig{variable-size-error-alpha}{Train and test error functions for
  several signals of varying number of points $d_i$ and length in base
  pairs $l_i$. The penalty contains a term $l_i^\alpha \sqrt{d_i}$.}

\newpage
\section{Optimal penalties for the fused lasso signal approximator}

In the previous sections, we used theoretical arguments and simulation
experiments to determine the optimal penalties for cghseg. In this
section, we demonstrate that the same approach can be used to find
optimal penalties for another model, the Fused Lasso Signal
Approximator (FLSA).

We used the \verb|flsa| function in version 1.03 of the flsa
package from CRAN to calculate the FLSA \citep{fused-lasso-path}. Let
$x\in\RR^d$ be the noisy copy number signal for one chromosome. The
FLSA solves the following optimization problem:
\begin{equation}
  \label{eq:flsa}
\argmin_{\mu\in\RR^d} 
\frac 1 2 \sum_{j=1}^d (x_j-\mu_j)^2
+\lambda_1\sum_{j=1}^d|\mu_j|
+\lambda_2\sum_{j=1}^{d-1}|\mu_j-\mu_{j+1}|.
\end{equation}

First, we take $\lambda_1=0$ since we are concerned with breakpoint
detection, not signal sparsity. In this section, our aim is to
determine a parameterization for $\lambda_2$ that we will be able to
find similar breakpoints in signals of varying sampling density.

We use the same setup that we used to determine optimal penalties for
cghseg, as described in Section~\ref{variable_density} and shown again
in Figure~\ref{fig:variable-density-signals-flsa}.

\begin{figure}[h]
  \centering
  \includegraphics[width=\linewidth]{figures/variable-density-signals}
  \caption{Simulated signals with different
  sampling density.}
  \label{fig:variable-density-signals-flsa}
\end{figure}

\newpage

In particular, for every signal $i\in\{1,\dots,n\}$, let
$y_i\in\RR^{d_i}$ be the noisy signals, sampled at positions
$p_i\in\mathcal X^{d_i}$. To find an optimal penalty for these data,
first let $\lambda_2 = \lambda d_i^\alpha$. For each signal $i$,
exponent $\alpha\in\RR$, and tradeoff parameter $\lambda\in\RR^+$, we
define the optimal smoothing as
\begin{equation}
  \label{eq:flsa_lambda}
  \hat y^{\lambda,\alpha}_i = 
\argmin_{\mu\in\RR^{d_i}} 
\frac 1 2 ||y_i-\mu||_2^2
+\lambda d_i^\alpha \sum_{j=1}^{d_i-1} |\mu_j - \mu_{j+1}|.
\end{equation}
% \begin{equation}
%   \label{eq:flsa_lambda}
%   \hat y^{\lambda,\alpha}_i = 
% \argmin_{\mu\in\RR^{d_i}} 
% \frac 1 2 ||y_i-\mu||_2^2
% +\lambda d_i^\alpha \TV_i(\mu),
% \end{equation}
% with the total variation $\TV_i:\RR^{d_i}\rightarrow \RR^+$ defined as
% \begin{equation}
%   \label{eq:TV}
%   \TV_i(x) = \sum_{j=1}^{d_i-1} |x_j - x_{j+1}|.
% \end{equation}

Then, we define the breakpoint detection error as a function of the
breaks in the smoothed signal:
\begin{equation}
  \label{eq:flsa_e_i_alpha}
  E_i^\alpha(\lambda) = 
E^B_{\text{exact}}
\left[
\phi\left(
\hat y^{\lambda,\alpha}_i,p_i
\right)
\right],
\end{equation}
where the breakpoint function $\phi$ is defined in
(\ref{eq:breaks_phi}) and the error $E_{\text{exact}}^B$ is defined
(\ref{eq:exact_breakpoint_cost}).

We plot $E_i^\alpha$ for 2 signals $i$ and several penalty exponents
$\alpha$ in Figure~\ref{fig:variable-density-berr-flsa}. Note that the
functions appear to align when $\alpha=1$.

\fig{variable-density-berr-flsa}{Model complexity breakpoint error
  functions $E_i^\alpha$.}

\newpage

To evaluate which penalty parameter $\alpha$ results in optimal
fitting and learning, we use $E^*$ and TestErr as defined in
(\ref{eq:lerr_train}) and (\ref{eq:lerr_test}). These functions
are plotted in Figure~\ref{fig:variable-density-error-alpha-flsa}, and
suggest that a value of $\alpha=1$ is optimal.  This analysis suggests
that taking $\lambda_2=\lambda d_i$ is optimal for breakpoint
detection using FLSA. This agrees with the observation in
Chapter~\ref{chapter:bams} that the flsa.norm penalty with a $d_i$
term works better than the un-normalized flsa penalty.

We conclude by noting that this procedure could also be applied to
find penalties for FLSA that depend on estimated signal noise~$\hat
s_i^2$ and length in base pairs~$l_i$. However, we did not pursue this
since FLSA does not work as well as cghseg in practice on real data,
as shown in Chapter~\ref{chapter:bams}.

\fig{variable-density-error-alpha-flsa}{Train and test error as a
  function of penalty exponent $\alpha$. The penalty has a term for
  the number of points sampled $d_i^\alpha$.}


\newpage

\section{Application to real data}

In Sections~\ref{variable_density}-\ref{variable_size}, we found
optimal cghseg penalties for data with varying sampling density,
scale, and length. In Section~\ref{combining_penalties}, we also
demonstrated that these results can be combined. This analysis
suggests the following penalty, for every signal $i$:
\begin{equation}
  \label{eq:composite_penalty}
  k_i(\lambda) = 
  \argmin_k
  \lambda k \hat s_i^2 \sqrt{d_i/l_i}  +
  ||y_i - \hat y_i^k||^2_2
\end{equation}

In Table~\ref{tab:penalty-real-data}, we report results of using the
suggested penalties on the neuroblastoma data set. The normalizations
suggested by the analysis of simulations do not improve breakpoint
detection error in the neuroblastoma data set. This observation
suggests that distribution that generates the real data is more
complex than the simple simulation model considered in this chapter.

Practically speaking, we still would like to find a penalty with
optimal breakpoint detection in real data. So, in the next chapter, we
introduce a new method that uses interval regression to exploit the
breakpoint annotations and to learn an optimal penalty from real data.

\tab{penalty-real-data}{Breakpoint detection error of cghseg on the
  neuroblastoma data set, with 1 row for each penalty. The exponent of
  the points $d_i$, length $l_i$, and variance $\hat s_i$ terms in the
  penalty is shown with the train and test error in percent. }

%\bibliographystyle{abbrvnat}
\bibliography{refs}

\end{document}
